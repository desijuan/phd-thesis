%
% SECTION 1 - FIRST DEFINITONS
%

% GROUPOIDS
A \emph{groupoid} $G\tto M$ is a small category in which every arrow is invertible.
It has a set of objects $M=\{x,y,z,\dots\}$ and a set of arrows $G=\{g,h,\dots\}$.
Each arrow $g$ in $G$ has two objects associated to it, the source $s(g)$ and the target $t(g)$.
\[ s,t\colon G\to M  \quad s(y\xfrom{g} x) = x  \quad t(y\xfrom{g} x) = y \]
There is an associative composition law that sends each pair of arrows $g$ and $h$ with $s(h) = t(g)$ to an arrow $hg$ from $s(g)$ to $t(h)$.
Here, by $G\times_M G$ we mean the fiber product between the target and the source maps.
\[ \circ\colon G\times_M G \to G \quad (z\xfrom{h}y, y\xfrom{g}x) \mapsto (z\xfrom{hg}x) \]
For each object $x$ there is a unit arrow \smash{$x\xfrom{u_x} x$}, that satisfies $u_x g = g$ and $h u_x = h$ for every $g$ with $t(g) = x$ and every $h$ with $s(h) = x$.
\[ u\colon M \to G \quad x \mapsto u_x \]
The arrows are required to be invertible, so for each $y\xfrom{g} x$ there is another arrow in the opposite direction $g^{-1}$, satisfying $g^{-1}g = u_x$ and $gg^{-1} = u_y$.
\[ i\colon G \to G \quad g \mapsto g^{-1} \]

% LIE GROUPOIDS
A \emph{Lie groupoid} is a groupoid in which the sets of objects and arrows are manifolds, the structural maps $s$, $t$, $\circ$, $u$ and $i$ are all smooth and the source and target maps are surjective submersions.

As the source and target are left inverses of the unit map, the manifold of objects is an embedded submanifold of the arrows.
With this in mind we will often write $G$ instead of $G\tto M$ to refer to the Lie groupoid in question.
Also note that since the source and target are submersions the domain of the multiplication $G\times_M G$ is a good fiber product (see \cite[\textsection 2.2]{dh13}).

% MORPHISMS OF LIE GPDS
A \emph{morphism} of Lie groupoids $\phi\colon (G\tto M)\to (H\tto N)$ is a smooth functor.
It consists of two smooth maps $M\to N$ and $G\to H$, both denoted again with $\phi$, which commute with the structural maps, that is $s(\phi(g)) = \phi(s(g))$, $t(\phi(g)) = \phi(t(g))$, $\phi(hg) = \phi(h)\phi(g)$, $\phi(u_x) = u_{\phi(x)}$ and $\phi(g^{-1}) = \phi(g)^{-1}$ for every arrow $g\in G$ and every object $x\in M$.
\[
\phi =
\begin{cases}
  G\xto{\phi} H \\
  M\xto{\phi} N
\end{cases}
\]

% EQUIVALENCES
Given two morphisms of Lie groupoids $\phi$ and $\psi$ from $G\tto M$ to $H\tto N$, an \emph{equivalence} $\alpha\colon\phi\Rightarrow\psi$ is a smooth map $\alpha\colon M\to H$ that sends a point $x$ in $M$ to an arrow \smash{$\psi(x)\xfrom{\alpha_x}\phi(x)$}, verifying $\alpha_y \phi(g) = \psi(g) \alpha_x$ for every \smash{$y\xfrom{g} x$} in $G$.
\begin{equation}
\begin{tikzcd}[sep=small]
  \phi(x) \ar{r}{\phi(g)} \ar{d}{\alpha_x} & \phi(y) \ar{d}{\alpha_y} \\
  \psi(x) \ar{r}{\psi(g)} & \psi(y)
\end{tikzcd}
\end{equation}

% ISOTROPY GROUPS
Let us denote by $G(y,x)$ the subset of $G$ consisting of all arrows from $x$ to $y$.
Given an $x\in M$, the \emph{isotropy group} of $G$ at $x$, denoted by $G_x$, is the set $G(x,x)$ of arrows starting and ending at $x$,
%
% ORBITS
and the \emph{orbit} $O_x\subset M$ is the subset of all objects that are connected to $x$ by an arrow.

\medskip\noi \red{DIBUJO. Explicar mejor $\uparrow$.}

% G(y,x)\subset G ARE REGULAR SUBMANIFOLDS
\begin{prop}[\cite{mm03, mack05}]
Given $G\tto M$ and $x,y\in M$, the subset $G(y,x)\subset G$ is an embedded submanifold. In particular $G_x$ is a Lie group. The orbit $O_x\subset M$ is a (maybe not embedded) submanifold in a canonical way. \red{the s-fiber is a principal bundle}
\end{prop}

% ORBIT SPACE M/G
The \emph{orbit space} $M/G$ is set of orbits with the quotient topology
%
% NORMAL SPACES
and the \emph{normal space} $N_xO\subset T_xM$ is the quotient of vector spaces $T_xM/T_xO$.

% ANCHOR
The map $(t,s)\colon G\to M\times M$ is called the \emph{anchor map} of the groupoid.
%
% PROPER AND ÉTALE GPDS
A Lie groupoid with a proper anchor map is a \emph{proper} groupoid.
If the manifolds of objects and arrows $M$ and $G$ have the same dimension, the Lie groupoid $G\tto M$ is said to be \emph{étale}.
%
% TRANSITIVE GPDS
A Lie groupoid is \emph{transitive} if it has a single orbit, or in other words, if for every pair of objects there is an arrow  between them.
