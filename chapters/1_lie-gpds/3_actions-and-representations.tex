%
% SECTION 3 - ACTIONS AND REPRESENTATIONS
%

\section{Actions and representations}\label{sec:actions-and-representations}

% ACTIONS
Let $G\tto M$ be a Lie groupoid, $A$ a manifold and $p\colon A\to M$ a map of constant rank.
We denote by $A_x=p^{-1}(x)$ the fiber of $p$ over the point $x$ of $M$.
A (left) \emph{groupoid action} $\theta\colon (G\tto M)\acts (A\to M)$ is a smooth map
\[ \theta\colon G\times_M A\to A \qquad (y\xfrom{g}x, a)\mapsto \theta_g(a) \]
such that $p(\theta_g(a)) = t(g)$, $\theta_{\id}=\id$ and $\theta_{hg}=\theta_h\theta_g$.
Here, by $G\times_M A$ we mean the fiber product between the source map $s\colon G\to M$ and the map $p\colon A\to M$, which is often called the \emph{moment map} of the action.
For each arrow $y\xfrom g x$ there is a diffeomorphism $\theta_g\colon A_x\to A_y$, so, an action is a way of realizing the arrows of the groupoid $G\tto M$ as symmetries of the family of fibers of $p$.

% REPRESENTATIONS
Given a Lie group $G\tto M$ and a vector bundle $E\to M$, a \emph{representation} $\rho\colon (G\tto M)\acts (E\to M)$ is an action in which the maps $\rho_g\colon E_x\to E_y$ are all linear.
% It is given by a smooth map
% $$ \rho\colon G\times_M E\to E \qquad (y\xfrom{g}x, e)\mapsto \rho_g(e) $$
% such that $\pi(\rho_g(e)) = t(g)$, the map $\rho_g\colon E_x\to E_y$ is linear, $\rho_{\id}=\id$ and $\rho_h\rho_g = \rho_{hg}$.
%
% RANK
The \emph{rank} of $\rho$ is defined as the rank of the vector bundle $E\to M$.

%EX: MANIFOLDS
\begin{example}
Let $M$ be a manifold and $E\to M$ a vector bundle.
There is only one representation of the unit groupoid $M\tto M$ on $E\to M$, the trivial one.
So, the representations of a unit groupoid $M\tto M$ are in one-to-one correspondence with the vector bundles over $M$.
\end{example}

%EX: LIE GROUPS
\begin{example}
A vector bundle over the point $V\to *$ is a vector space.
Representations of Lie Groups $(G\tto *)\acts (V\to *)$ are vector spaces equipped with an usual group representation.
\end{example}

%EX: PAIR GPD
\begin{example}
Let $\rho$ be a representation of a pair groupoid $M\times M\tto M$ on a vector bundle $E\to M$.
For any fixed $x_0\in M$ the map $E\to M\times E_{x_0}$, $e\mapsto (x, \rho_{(x_0,x)}(e))$ is a trivialization of $E$.
So, if a vector bundle $E\to M$ admits a representation of the pair groupoid $M\times M\tto M$, then it has to be trivializable.
Conversely, given a global trivialization $\varphi\colon E\to M\times\RR^k$ of $E\to M$, the map $M\times M\times_M E \to E$, $((y,x),x))\mapsto \varphi^{-1}(y,\varphi_2(e))$ is a representation of the pair groupoid $M\times M\tto M$.
\end{example}

In example \ref{ex:action-gpd} we mentioned how an action $G\acts M$ of a Lie group on a manifold gives rise to an action groupoid $G\times M\tto M$.
This can be generalized to the context of Lie groupoids.
Let $\theta\colon (G\tto M)\acts (A\to M)$ be a Lie groupoid action, its \emph{action groupoid} \(G\times_M A\tto A\) is a Lie groupoid over $A$ whose arrows are \smash{$\theta_g(a) \xfrom{(g,a)} a$},
the source is $s(g,a) = a$, the target $t(g,a) = \theta_g(a)$ is given by the action of $G$, the composition, unit and inverse maps are induced by those of $G$.
We say that the action $G\acts A$ is \emph{free} if the action groupoid has no isotropy, and that the action is \emph{proper} if the map $G \times_M A \to A\times A$, $(g,a)\mapsto (\theta_g(a),a)$ is so.

% ACTION MAPS
Given an action of Lie groupoids $(G\tto M)\acts (A\to M)$, the moment map $A\to M$ and the projection $G\times_M A\to G$ define a morphisim of Lie groupoids of particular nature.
Given a constant rank map $p\colon A\to M$, a morphism of Lie groupoids $\phi\colon(H\tto A)\to (G\tto M)$ is an \emph{action map} if it induces a good pullback (see \cite{dh13} section 2.2) between the sources.
\begin{equation}
\begin{tikzcd}[sep = small]
  H \ar{r}{s} \ar{d}{\phi} & A \ar{d}{p} \\
  G \ar{r}{s} & M
\end{tikzcd}
\end{equation}

The projection \((G\times_M A\tto A)\to (G\tto M)\) associated to an action of Lie groupoids $(G\tto M)\acts (A\to M)$ is an action map.
Conversely, if $(H\tto A)\to (G\tto M)$ is an action map, then $H$ is diffeomorphic to $G\times_M A$ and the composition \smash{$G\times_M A\to H\xto t A$} is an action, it is called the \emph{underlying action}.
It is straightforward to check that these constructions are mutually inverse.

\begin{prop}[\cite{mack05}]
There is a one-to-one correspondence between left actions and action maps.
\end{prop}

% AS A MAP FROM G TO GL(E)
If $\rho\colon G\times_ME\to E$ is a representation, for each arrow $y\xfrom g x$ there is an isomorprism of vector bundles $\rho_g\colon E_x\to E_y$, which is an element of the General Linear Groupoid described in example \ref{ex:gle}.
So, one could alternatively define a representation as a morphism of Lie groupoids from $G$ the General Linear Groupoid $GL(E)$ covering the identity of $M$.
This correspondence is of particular interest for us, since one of the main subjects of the present thesis is to generalize it to the context of Lie groupoids and vector bundles over them REF.

\begin{prop}\label{repr}
There is a one-to-one correspondence between representations of $G\tto M$ over $E\to M$ and Lie groupoid maps from $G\tto M$ to the General Linear Groupoid $GL(E)$
% \[ rho\colon (G\tto M)\to (GL(E)\tto M) \qquad g \mapsto \rho_g \]
that are the identity over $M$.
\end{prop}

\begin{proof}
Given a representation $\rho\colon G\times_ME\to E$, let us consider the morphism of groupoids $\psi\colon(G\tto M)\to(GL(E)\tto M)$ that is the identity over $M$ and sends an arrow \smash{$y\xfrom{g} x$} to the linear isomorphism \smash{$\rho_g\colon E_x\to E_y$}.
To see that the morphism $\psi$ is smooth, let us pick a chart \smash{$\varphi_{ji}\colon GL(E)(U_j)(U_i)\to U_j\times U_i\times\RR^{k^2}$} of $GL(E)$ (see example \ref{ex:gle}) and consider the composition \smash{$\varphi_{ji}\res{\psi}{G(U_j,U_i)}\colon G(U_j,U_i)\to U_j\times U_i\times\RR^{k^2}$}.
By looking at the last factor we have a map \smash{$A\colon G(U_j,U_i)\to \RR^{k^2}$} that assigns the matrix \smash{$A(g)=\varphi_j\rho_g\varphi_i^{-1}\in\RR^{k^2}$} to each arrow $g\in G(U_j,U_i)$.
Let $\{e_1,\dots,e_k\}$ be the canonical basis of $\RR^k$ and $\{e^1,\dots,e^k\}$ its dual basis.
For each $1\leq r,s\leq k$ the entry $rs$ of $A$ is $A{rs}=e^s\circ\varphi_j\circ\rho\circ\varphi_i^{-1}\circ i_{e_r}$,
which is a smooth function of $g$.
Here, $i_{e_r}\colon G\to G\times\RR^k$ is $g\mapsto(g,e_r)$,
Therefore, \smash{$A\colon G(U_j,U_i)\to \RR^{k^2}$} is smooth and so is $\psi\colon G\to GL(E)$.
This shows that we have a well defined morphism of Lie groupoids $\psi\colon(G\tto M)\to(GL(E)\tto M)$.

Conversely, if we have a morphism of Lie groupoids $\psi\colon(G\tto M)\to(GL(E)\tto M)$, let $\rho\colon G\times_ME\to E$ be the map given by $\rho(g,e)=\psi_g(e)$.
As $\psi$ is a morphism of groupoids, $\rho$ is an action.
To see that it is smooth, note that $\rho=\text{ev}\circ(\psi\times \text{id}_E)$, where $\text{ev}\colon GL(E)\times_M E\to E$ is the evaluation $\text{ev}(\xi,e)=\xi(e)$.
\end{proof}

If one tries to carry on the characterization of Proposition \ref{repr}, but for actions in general instad of representations, the resulting space is infinite dimensional.
\red{Should be developed. Explain the set-theoretic situation and the smooth aspects.}

% MORPHISMS OF REPRESENTATIONS
If\nota{subir esto?} $\rho_1\colon (G\tto M)\acts (E_1\to M)$ and $\rho_2\colon (G\tto M)\acts (E_2\to M)$ are representations of the same groupoid $G\tto M$, a \emph{morphism} $\eta\colon \rho_1 \Rightarrow \rho_2$ between them is a morphism of vector bundles $\eta\colon E_1 \to E_2$ such that $\eta_y\rho_1(g) = \rho_2(g)\eta_x$ for every arrow $y\xfrom{g}x$ of $G$.

% PULLBACK OF REPRESENTATIONS
Let $\rho\colon (G\tto M)\acts (E\to M)$ be a representation.
Given another Lie groupoid $H\tto N$ and morphism $\phi\colon (H\tto N) \to (G\tto M)$, we can consider the \emph{pull-back} of $\rho$ via $\phi$, denoted by \(\phi^*(\rho)\colon (H\tto N)\acts (\phi^*E\to N)\).
This is a representation of $H\tto N$ on the pull-back vector bundle $\phi^*E\to N$ that sends an arrow \smash{$w\xfrom h z$} to the linear isomorphism $\rho(\phi(h))\colon E_{\phi(w)}\to E_{\phi(z)}$.

% EX: VECTOR BUNDLES AS REPRESENTATIONS
\begin{example}
In example \ref{ex:vectorBundle} we observed that, by picking a trivializing open cover $\U=\{U_i\}$ of the base, a vector bundle of rank $k$ over $M$ can be seen a morphism of Lie groupoids $\theta\colon M_\U \to GL_k$.
Considering the pull-back of the identity representation of $GL_k$ on $\RR^k$ via the morphism $\theta$, we can also think of a vector bundle, or a $GL_k$-cocycle, as a representation of the \v Cech groupoid $M_\U$ on the pull-back vector bundle $\theta^*(\RR^k) = \big( \coprod U_i\times \RR^k \to \coprod U_i \big)$.
\end{example}

\red{Write example on submersion groupoids, it works exactly as the open cover case, it is called ``descent theory'', read wikipedia article about it.}

% % En realidad esta proposición son dos:
% % 1) El pull-back de representaciones es funtorial
% % 2) Si $\phi_1$ y $\phi_2$ son morfismos de fibrados naturalmente equivalentes, entonces $\phi_1^* = \phi_2^*$.
% \begin{prop}\label{iso}
% Let $\phi_1$ and $\phi_2$ be morphisms of Lie groupoids from $G'\tto M'$ to $G\tto M$ and $\rho_1\colon (G\tto M)\to (GL(E_1)\tto M)$ and $\rho_2\colon (G\tto M)\to (GL(E_2)\tto M)$ two representations of $G\tto M$.
% % If $\phi_1\simeq\phi_2$ and $\rho_1\simeq\rho_2$, then $\phi_1^*\rho_1 \simeq \phi_2^*\rho_2$.
% If the morphisms $\phi_1$ and $\phi_2$ are naturally equivalent and the representations $\rho_1$ and $\rho_2$ are isomorphic, then so are $\phi_1^*\rho_1$ and $\phi_2^*\rho_2$.
% \end{prop}

% \begin{proof}
% Let $\eta\colon E_1\to E_2$ be an isomorphism of vector bundles providing the isomorphism $\rho_1\simeq\rho_2$ and let $\alpha$ be a natural equivalence $\phi_1\simeq\phi_2$.
% Given an $x'\in M'$, let $\mu_{x'}\colon \left(\phi_1^*E_1\right)_{x'} \to \left(\phi_2^*E_2\right)_{x'}$ be the isomorphism of vector spaces defined by $\mu_{x'} = \eta_{\phi_2(x')}\rho_1(\alpha_{x'})$.
% This gives us a section of the fiber bundle $\Iso\left(\rho_1^*E_1,\rho_2^*E_2\right)$ such that $\mu_{y'}\rho_1(\phi_1(g')) = \rho_2(\phi_2(g'))\mu_{x'}$ for every arrow $g'\colon x'\to y'$ of $G'$.
% \end{proof}

% \begin{prop}\label{morita}
% If $\phi\colon (G\tto M) \to (G'\tto M')$ is a Morita morphism such that the map on the objects $\phi\colon M\to M'$ is a surjective submersion, then the pull-back induces a bijection between the sets of isomorphism classes of representations of $G\tto M$ and of $G'\tto M'$.
% \end{prop}

% \begin{proof}
% As the morphism $\phi$ is a surjective submersion on the objects, choosing local sections we can obtain an open cover $\U'$ and a morphism $\sigma\colon G'_{\U'} \to G$ such that $\pi'_{\U'} = \phi\sigma$.
% Consider the cover $\U$ of $M$ that has all the preimages via $\phi$ of all the open sets $U'$ in $\U'$.
% We have the following commutative diagram.
% \begin{equation*}
% \begin{tikzcd}
%  G_\U \ar{r}{\pi_\U} \ar{d}[swap]{\phi_\U} & G \ar{d}{\phi} \\
%  G'_{\U'} \ar{r}[swap]{\pi'_{\U'}} \ar{ur}{\sigma} & G'
% \end{tikzcd}
% \end{equation*}
% Observing that $\sigma^*\phi^* = \pi'^*_{\U'}$ and $\phi_\U^*\sigma^* = \pi_\U^*$ we get that if $\pi'^*_{\U'}$ and $\pi^*_\U$ induce bijections between the respective sets of representations, then so does $\phi^*$.
% Hence, it is enough to prove that the result holds for a morphism of Lie groupoids of the form $\pi_\U\colon G_\U \to G$.

% If $\rho_1\colon (G\tto M) \to (GL(E_1)\tto M)$ and $\rho_2\colon (G\tto M) \to (GL(E_2)\tto M)$ are two representations of $G\tto M$ such that the pull-backs $\pi_\U^*(\rho_1)$ and $\pi_\U^*(\rho_2)$ are isomorphic, let us say, via an isomorphism of vector bundles $\mu\colon\pi_\U^*(E_1) \to \pi_\U^*(E_2)$.
% Putting $\eta_i(x) = \mu_{(x,i)}$ we have a family of isomorphisms of vector bundles $\eta_i\colon \res{E_1}{U_i} \to \res{E_2}{U_i}$.
% Note that $\eta_i(x) = \pi_\U^*\rho_2(u_x,i,j) \, \mu_{(x,i)} = \mu_{(x,j)} \, \pi_\U^*\rho_1(u_x,i,j) = \eta_j(x)$ for every $x \in U_{ij}$.
% This allows us to glue the $\eta_i$ all together to get a morphism of vector bundles $\eta\colon E_1 \to E_2$ that gives us the desired isomorphism of representations between $\rho_1$ and $\rho_2$.
% This proves the injectivity of $\pi_\U^*$.

% For the surjectivity, given a representation $\tau\colon (\coprod_{i,j} G(U_i,U_j) \tto \coprod_i U_i) \to (GL(F)\tto \coprod_i U_i)$ of $G_\U$, let us consider the vector bundle $E = ( \coprod_i F_{(x,i)} ) / _\sim$, where we are identifying a $v$ in the fiber $F_{(x,i)}$ with $\tau(u_x,i,j) \, v$ in $F_{(x,j)}$ for every $j$ such that $x$ also belongs to $U_j$.
% The representation $\rho\colon (G\tto M) \to (GL(E)\tto M)$ of $G$ that sends an arrow $g\colon x\to y$ to the isomorphism $\rho\colon E_x \to E_y$ induced by $\tau(g,i,j)\colon F_{(x,i)} \to F_{(x,j)}$ where $i$ and $j$ are any such that $x$ is in $U_i$ and $y$ is in $U_j$, satisfies that $\pi_\U^*(\rho) = \tau$.
% \end{proof}

% \begin{thm}
% The pull-back of representations induces a natural bijection between the set of maps of stacks from a manifold $M$ to the classifying stack of the general linear groupoid $BGL(k,\CC)$ and the set of isomorphism classes o rank $k$ vector bundles over $M$.
% \end{thm}

% \begin{proof}
% Let us use $G\tto *$ as an abbreviation for the general linear group $GL(k,\CC)$.
% If we are given a map of stacks from $M$ to $BG$, let us say represented by a fraction
% \begin{equation*}
% \begin{tikzcd}
%  (M\tto M) & (H\tto N) \ar{l}[sloped, below]{\sim}[sloped, above]{\phi} \ar{r}{\psi} & (G\tto *) ,
% \end{tikzcd}
% \end{equation*}
% we can pull-back the identity representation of $G\tto *$ to obtain a representation of $H\tto N$, let us call it $\tau$.
% As the map $\phi$ is Morita and a surjective submersion on the objects, by proposition \ref{morita} we know that there exists a unique (modulo isomorphism) representation $\rho\colon (M\tto M) \to (GL(E)\tto M)$ such that that $\phi^*\rho$ and $\tau$ are isomorphic.
% Propositions \ref{iso} and \ref{morita} together with the observations made in example \ref{rep-unit} assure us that we have a well defined mapping $\psi/\phi \mapsto [E\to M]$, that does not depend on the choice of the representative $(\psi,\phi)$.
% Let us denote this mapping with $\psi/\phi \mapsto (\psi/\phi)^*(1) = [E\to M]$.

% By lemma \ref{representative}, we can choose a representative of the form of the form $\theta/\pi_\U$ for some open cover $\U$ of $M$.
% To have an isomorphims class of vector bundles $[E\to M]$ is the same as an equivalence class of cocycles $[(\U,\{\theta_{ij}\})]$, and this is the same as a map of stacks $\theta/\pi_\U$.

% \noi FALTA TERMINAR.
% \end{proof}
