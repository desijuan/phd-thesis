%
% SECTION 4 - NERVE AND COHOMOLOGY
%

\section{Nerve and cohomology}\label{sec:nerve-and-cohomology}

% NERVE
The \emph{nerve} of a Lie groupoid $G\tto M$, denoted $NG$, is the simplicial manifold whose $k$-simplices $N_kG$ are the strings of composable arrows $(g_k,\dots,g_1)$ such that $s(g_{i+1}) = t(g_i)$ for all $1\leq i\leq k-1$.
So, $N_0G = M$, $N_1G = G$, $N_2G = G\times_M G$ and so on.
\[ (g_k,\dots,g_1) \colon x_k \leftarrow \dots \leftarrow x_i \xfrom{g_i} x_{i-1} \leftarrow \dots \leftarrow x_0 \]
The \emph{face maps} $d_i\colon N_kG \to N_{k-1}G$ are $d_1=s$ and $d_0=t$ for $k=1$, and
\begin{equation}
d_i(g_k,\dots,g_1) =
\begin{cases}
  (g_k,\dots,g_2) & i = 0 \\
  (g_k,\dots,g_{i+1}g_i,\dots,g_1) & 1 \leq i \leq k-1 \\
  (g_{k-1},\dots,g_1) & i = k
\end{cases}
\end{equation}
for $k>1$.
The \emph{degeneracies} $s_j\colon N_kG \to N_{k+1}G$ are
\begin{equation}
s_j(g_k,\dots,g_1) =
\begin{cases}
  (g_k,\dots,g_1,u_{s(g_1)}) & j = 0 \\
  (g_k,\dots,g_{j+1},u_{t(g_j)},g_j,\dots,g_1) & 1 \leq j \leq k . \\
\end{cases}
\end{equation}
% A $k$-simplex is \emph{degenerated} if it is in the image of one of the degeneracies.

% DIFFERENTIABLE COHOMOLOGY
Considering the functions on the manifolds of $k$-simplices and the differentials
\[ \delta\colon C^\infty(N_kG)\to C^\infty(N_{k+1}G) \qquad \delta = \sum_{i=0}^{k+1} (-1)^{k+1-i} \, d_i^* \]
we get the \emph{differentiable complex} $(C^\infty(NG), \delta)$.
\[ C^\infty(NG)\colon \quad C^\infty(M) \xto{\delta} C^\infty(G) \xto{\delta} C^\infty(G\times_M G) \xto{\delta} \cdots \]
Its cohomology $H_\text{diff}^*(G)$  is the \emph{differentiable cohomology} of the Lie groupoid $G\tto M$.

% NON-REDUCED BOTT-SHULMAN COMPLEX
For each $k$, we can also consider the $l$-forms on the manifolds of $k$-simplices together with the maps
\[ \delta\colon \Omega^l(N_kG)\to \Omega^l(N_{k+1}G) \qquad \delta = \sum_{i=0}^{k+1} (-1)^{i+l} \, d_i^* \]
to get the chain complexes $(\Omega^l(NG), \delta)$.
\[ \Omega^l(NG)\colon \quad \Omega^l(M) \xto{\delta} \Omega^l(G) \xto{\delta} \Omega^l(G\times_M G) \xto{\delta} \cdots \]
If we also take the De Rham cohomology into acount we get a double complex, the (non-reduced) \emph{Bott-Shulman complex}:
\begin{equation}
\begin{tikzcd}[sep = small]
  \vdots & \vdots & \vdots \\
  \Omega^2(M) \ar{u} \ar{r}{\delta} & \Omega^2(G) \ar{u} \ar{r}{\delta} & \Omega^2(N_2G) \ar{u} \ar{r} & \cdots \\
  \Omega^1(M) \ar{u}{d} \ar{r}{\delta} & \Omega^1(G) \ar{u}{d} \ar{r}{\delta} & \Omega^1(N_2G) \ar{u}{d} \ar{r} & \cdots \\
  C^\infty(M) \ar{u}{d} \ar{r}{\delta} & C^\infty(G) \ar{u}{d} \ar{r}{\delta} & C^\infty(N_2G) \ar{u}{d} \ar{r} & \cdots
\end{tikzcd}
\end{equation}
%
% REDUCED BOTT-SHULMAN COMPLEX
The degenerated simplices, that is, the ones that are in the image of the degeneracies, will not be of interest for us.
So, we will replace the spaces of $l$-forms $\Omega^l(N_kG)$ with the ones that are zero on the degenerated simplices.
\begin{equation}
K^{k,l}(G\tto M) =
\begin{cases}
    \Omega^l(M) & k = 0 \\
    \bigcap_{i=0}^{k-1} \Ker\left( s_i^*\colon \Omega^l(N_kG) \to \Omega^l(N_{k-1}G) \right) & k > 0
\end{cases}
\end{equation}
The resulting complex $(K^{*,*}(G),\delta,d)$ is knwown as the \emph{reduced Bott-Shulman complex} of the Lie groupoid $G$.
%
% COHOMOLOGY OF LIE GROUPOIDS
The cohomology of $G\tto M$ is defined as the cohomology of the total complex of $(K^{*,*}(G), d, \delta)$.
\[ H^*(G) = H^*(\text{Tot}(K^{*,*}(G))) \]

It is worth noting that both constructions, the reduced and the non-reduced version of the Bott-Shulman, yield equivalent complexes.
Since it is more convenient to cross out the degenerated simplices, we will prefer the reduced version over the non-reduced one.

\begin{example}
Let $M\tto M$ be a unit groupoid, its the Bott-Shulman complex has zeros everywhere except for the first column.
So, the Bott-Shulman complex of a manifold is the De Rham complex.
\end{example}

\begin{example}
Let $M_\U$ be the \v{C}ech grupoid of a manifold $M$ with an open cover $\U$ that we presented in example \ref{ex:cover}.
The (reduced) Bott-Shulman complex of $M_\U$ is the \v{C}ech-de Rham complex (see \cite[chapter 2]{bott-tu}).
\end{example}

\begin{example}
Let $G\tto *$ be a Lie group, then $N_0G=*$, $N_1G=G$, $N_2G=G\times G$ and so on.
In this case the differentiable cohomology is the group cohomology of $G$, and the Bott-Shulman complex ???
\end{example}

\begin{example}
Let $(G\tto M)\acts(A\to M)$ be an action of Lie groupoids and let $G\times_M A\tto A$ be the corresponding action groupoid. ???
\end{example}

% ME GUSTARIA ENCONTRAR LA FORMA DE PONER ESTO
% \begin{prop}[Grothendieck]
% A\nota{acá hay que\\ definir un montón\\ de cosas antes} simplicial object $X_\bullet \in sC$ is isomorphic to the nerve of a Lie groupoid if and only if the maps
% $$ \lambda_i^k\colon X_k\to \Lambda_i^k $$
%   are covers for $k=1$ and isomorphisms for $k>1$.
% \end{prop}




% % DIFFERENTIAL COHOMOLOGY WITH COEFFICIENTS IN A REPRESENTATION
% Let us consider the maps
% \begin{align*}
% G_k &\xto{s} M \\
% (g_1,\dots,g_k) &\mapsto x_0 = s(g_1) \text{.}
% \end{align*}
% Given a vector bundle $E$ over $M$, let $C(G_k,E)$ be the vector space of sections of the pull-back $s^*E$.
% An element of $C(G_k,E)$ is a smooth map $\sigma\colon G_k\to E$ such that $\sigma(g_1,\dots,g_k)$ belongs to the fiber $E_{x_0}$.
% Let $C(G,E)$ be the direct sum $\bigoplus_{k\geq 0} C(G_k,E)$.
% If $\rho\colon G\to GL(E)$ is a representation of $G\tto M$ we can consider the maps
% $\delta_\rho\colon C(G_k,E)\to C(G_{k+1},E)$
% given by
% \begin{equation*}
% \delta_\rho(\sigma)(g_1,\dots,g_{k+1}) = \rho_{g_1}^{-1} d_0^*(\sigma)(g_1,\dots,g_{k+1}) + \sum_{i=1}^{k+1} (-1)^i \, d_i^*(\sigma)(g_1,\dots,g_{k+1}) \text{.}
% \end{equation*}
% \begin{lemma}
% The mapping $\rho \mapsto \delta_\rho$ induces a bijection between the representations $\rho\colon G\to GL(E)$ and the degree $1$ differential operators $\delta_\rho\colon C(G,E)\to C(G,E)$ with square zero.
% \end{lemma}

% Observe that when $E$ is the trivial line bundle $M\times\CC$ and $\rho$ is the trivial representation, we have that $(C(G,E),\delta_\rho)$ is the complex $(C^\infty(G_*),\delta)$ defined above.

% Given a vector bundle $E\to M$ and a representation $\rho\colon G\to GL(E)$, we define the \emph{differential cohomology with coefficients} in the representation $\rho$, denoted $H^*(G,E)$, to be the cohomology of the complex $(C(G,E),\delta_\rho)$.\\

% % COADJOINT REPRESENTATION OF A LIE GROUP
% Let us breafly recall the coadjoint representation of a Lie group $G\tto *$, it is given by the map
% \begin{align*}
% G &\to GL(\g^*) \\
% g &\mapsto \text{ad}_g
% \end{align*}
% where $\text{ad}_g(\xi)(X_e) = \xi(d_e\rho_{g^{-1}}(X_e))$ for every $\xi\in \g^*$ and $X_e\in \g$, and $\rho_g$ is the conjugation by the element $g$ of $G$.

% BOTT MANAGED TO COMPUTE THE HORIZONTAL COHOMOLOGIES OF THE BOTT-SHULMAN COMPLEX
%   $H_\delta ^p(\Omega^q(G_*) \simeq H_\text{diff}^{p-q}(G, S^q(\mathfrak{g}^*))$

% BOTT'S SPECTRAL SEQUENCE: E_{pq}^1 => H^{p+q}(BG)

% IF G IS A COMPACT LIE GROUP, THEN H^{p+q}(BG) = 0 if p \neq q
%   AND H^{2p}(BG) = S^q(\mathfrak{g}^*)^G








