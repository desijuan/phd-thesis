%
% SECTION 3 - VB-MORITA MORPHISMS
%

% VB-MORITA
Given VB-groupoids $(\Gamma\tto E)\to (G\tto M)$ and $(\Gamma'\tto E')\to (G'\tto M')$, a morphism $\Phi\colon \Gamma\to\Gamma'$ is \emph{VB-Morita} if the morphism on the total spaces $\Phi\colon (\Gamma\tto E)\to (\Gamma'\tto E')$ is Morita.

% PULL-BACK
Given a vector bundle $E\to M$ and a smooth map $\psi\colon N\to M$, one can consider the pull-back $\psi^*E\to N$, which is the vector bundle over $N$ whose total space is the fiber product $E\times_M N$ between the map $\psi\colon N\to M$ and the projection $E\to M$.
\[\begin{tikzcd}[sep=1em]
  \psi^*E \ar{r} \ar{d} & E \ar{d} \\
  N \ar{r}{\psi} & M
\end{tikzcd}\]
Given a VB-groupoid $(\Gamma\tto E)\to (G\tto M)$ and a morphism of Lie groupoids $\phi\colon (H\tto N)\to (G\tto M)$,
let us consider the pull-backs $\phi^*\Gamma\to H$ and $\phi^*E\to N$ of the vector bundles $\Gamma\to G$ and $E\to M$ via the maps on the arrows $\phi\colon H \to G$ and the objects $\phi\colon N\to M$ respectively.
\[
\phi =
\begin{cases}
  H\xto{\phi} G \\
  N\xto{\phi} M
\end{cases}
\quad\leadsto\qquad
\begin{tikzcd}[sep=1em]
  \phi^*\Gamma \ar{r} \ar{d} & \Gamma \ar{d} \\
  H \ar{r}{\phi} & G
\end{tikzcd}
\quad
\begin{tikzcd}[sep=1em]
  \phi^*E \ar{r} \ar{d} & E \ar{d} \\
  N \ar{r}{\phi} & M
\end{tikzcd}\]
In \cite{bcdh16} it is proved that  $\phi^*\Gamma = (\phi^*\Gamma\tto\phi^* E) \to (H\tto N)$ is a VB-groupoid over $H\tto N$.
It is called the \emph{pull-back} or base-change of $\Gamma$ along $\phi$.
There is also an induced morphism of VB-groupoids $\phi^*\Gamma\to\Gamma$.
\[\begin{tikzcd}[sep=small]
  (\phi^*\Gamma\tto\phi^*E) \ar{r} \ar{d} & (\Gamma\tto E) \ar{d} \\
  (H\tto N) \ar{r}{\phi} & (G\tto M)
\end{tikzcd}\]

This construction defines a \emph{base-change} functor which preserves short exact sequences and duals.
\[ \phi^*\colon\mathrm{VB}(G\tto M) \to \mathrm{VB}(H\tto N) \]

% FIBER
Given an object $x\in M$, the \emph{fiber} of $\Gamma$ over $x$, denoted by $\Gamma_x$, is the VB-groupoid over the point $*\tto *$ defined as the base change of $\Gamma$ along the inclusion $*\xto{x}G$.

In an analogous way as with vector bundles over manifolds $E\to M$, in which a linear map $\phi\colon(E\to M)\to (F\to N)$ is an isomorphism if and only if the map on the base $\phi\colon M\to N$ is a diffeomorphism and the maps on the fibers $\phi_x\colon E_x\to F_x$ are isomorphisms, for a VB-morphism to be Morita it is enough to be Morita on the base and fiberwise Morita.

\begin{thm}[\cite{dho20}]\label{thm:VB-morita}
A VB-map $\Phi\colon\Gamma\to\Gamma'$ is VB-Morita if and only if the map on base $\varphi\colon (G\tto M)\to (G'\tto M')$ is Morita and the maps on the fibers $\Phi_x\colon \Gamma_x\to\Gamma'_{\phi(x)}$ al Morita for all $x\in M$.
\end{thm}

\begin{coro}\label{cor:quasi-iso}
Under the Grothendieck construction (see section \ref{sec:groth-corresp}), quasi-isomorphisms of representations up to homotopy correspond to VB-Morita morpisms.
\end{coro}

\begin{coro}
Given a VB-groupoid $(\Gamma\tto E)\to (G\tto M)$ and a morphism of Lie groupoids $\phi\colon (H\tto N)\to (G\tto M)$, the morphism induced in the pull-back $\phi^*\Gamma\to \Gamma$ is VB-Morita.
\end{coro}

When working with Lie groupoids, every morphism that admits an inverse up to isomorphism is Morita, but in general not every Morita morphism admits an inverse up to isomorphism.
The next proposition shows that in the context of VB-groupoids and linear morphisms both notions agree.

% VB-MORITA IFF INVERTIBLE UP TO ISOMORPHISM
\begin{prop}[\cite{dho20}]
A VB-morphism covering the identity is VB-Morita if and only if it admits an inverse up to isomorphism.
\end{prop}

To close the section, there is a characterization of Morita equivalence in the linear context.

% MORITA EQUIVALENT IFF SUM OF ACYCLIC
\begin{prop}[\cite{dho20}]
Two VB-groupoids $\Gamma$ and $\Gamma'$ over $G\tto M$ are Morita equivalent if and only if there are acyclic VB-groupouids $\Omega$ and $\Omega'$ over $G\tto M$ such that $\Gamma\oplus\Omega$ and $\Gamma'\oplus\Omega'$ are isomorphic.
\end{prop}
