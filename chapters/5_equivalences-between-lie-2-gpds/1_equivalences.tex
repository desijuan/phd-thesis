%
% SECTION 5.1 - EQUIVALENCES
%

When working with Lie groupoids (chapter \ref{ch:lie-gpds}), two fundamental concepts arise: Morita equivalences between them, and differentiable stacks.
These concepts were studied in some detail in chapter \ref{ch:stacks-and-syacky-vb}.
In some sense, one defines the other: a Morita equivalence is a morphism between Lie groupoids that induces an isomorphism between the corresponding stacks,
and conversely, a differentiable stack is defined as the class of a Lie groupoids modulo Morita equivalences.
In this section we explore and develop the analogous notions for Lie 2-groupoids, that of (Morita) equivalence between Lie 2-groupoids, and that of (differentiable) 2-stacks.
This is fundamental for our main theorem for we are going to classify stacky vector bundles over a fixed base, as defined in chapter 2, by morphisms of 2-stacks into the {\it generalized Grassmanian}, the 2-stack corresponding to the general linear 2-groupoid described in chapter 4.
Some approaches to these notion were developed in the literature, that we may summarize as follows: (1) an iterative approach to groupoids and morita equivalence, where equivalences are defined via pullbacks \cite{gs15},
and (2) a simplicial approach to higher groupoids, where the role of equivalences is played by the so-called hypercovers \cite{zhu09}.
For our purpose, we need to develop a more subtle theory, without the strictness of (1), but richer than (2). This is the main goal of the present chapter.

\bigskip

In section 1 we will give our original definition for (Morita) equivalences between Lie 2-groupoids, discuss some examples, and compare our approach with that of (1), building over our original results achieved in previous chapter.
Then, in section 2, we discuss the rudiments of the simplicial theory of higher Lie groupoids and hypercovers, discuss with more detail the case of 2-groupoids, and compare our definition with (2).
Finally, in section 3, we give the necessary constructions to make sense of a working definition of the category of differentiable 2-stacks.

%%%%%%%%%%%%%%%%

\section{Equivalences}

% setting notations

Recall that, given $G$ a Lie 2-groupoid, we have:
\begin{itemize}
\item an induced singular foliation $F_G\subset TG_0$ over the objects given by the orbits, and
\item for each pair of objects $x,y\in G_0$, we have a well-defined \emph{hom-groupoid},
$$G(y,x)=\big( G_2(y,x)\toto G_1(y,x) \big),$$
that is a regular Lie groupoid with abelian isotropy.
\end{itemize}
If $\phi: G\to H$ is a morphism of Lie 2-groupoids, then it must respect the orbits, and therefore the foliation $F_G\to F_H$, and it induces locally Lie groupoid morphisms
$$ \phi_{x,y}: G(y,x) \to H(\phi(y),\phi(x)). $$

% DEF: EQUIVALENCE
\begin{mydef}
A morphism of Lie 2-groupoids $\phi: G\to H$ is an \emph{equivalence} if it satisfies the following two conditions:
\begin{enumerate}
\item[E1)] $\phi_0: G_0\to H_0$ meets transversely every orbit, namely, $G_0\to H_0\to H_0/F_H$ is surjective and $T_xG_0\to N_{\phi(x)}H_0$ is surjective for every $x$;
\item[E2)] it is locally Morita, namely, for every $x,y$ the morphism of Lie 1-groupoids $\phi_{x,y}$ is Morita, as defined in \textsection \ref{sec:morita}.
\end{enumerate}
If in addition $G_0\to H_0$ is surjective submersion we
say that $\phi$ is a \emph{surjective equivalence}.
\end{mydef}

\noi OBS: Esta def tambien tiene sentido para lax functors.\\

We remark that if $\phi$ is an equivalence then it gives a bijection between the orbit spaces (pi0 argument explaining this), and it gives isomorphisms between the orbits of the hom-groupoids (pi1), and the isotropies of the hom-groupoids (pi2).

Let us discuss some very basic examples and we will see later more examples once we relate this definition with those in the literature.

% EX: LIE 1-GPDS
\begin{example}
If $G$ and $H$ are Lie 1-groupoids, seen as Lie 2-groupoids with trivial 2-cells, then $\phi$ is an equivalence if and only if it is Morita, as defined in \textsection \ref{sec:morita}.
In fact, in this case, the hom-groupoids are just discrete sets, being locally Morita is the same as being set-theoretically fully faithful, and we showed in Lemma \ref{ES-and-stFF} that a Lie groupoid morphisms that meets tranversely every orbit and is set-theoretically fully faithful must be Morita.
\end{example}

% Ex: Abelian Lie 2-groups
\begin{example}
Recall that a Lie 2-group is a Lie 2-groupoid with only one object.
Since there are only trivial objects, a morphism of Lie 2-grouops $\phi : (G_2\toto G_1\toto *) \to (H_2\toto H_1\toto *)$ is an equivalence if and only if the induced  morphism of Lie 1-groupoids $\phi : (G_2\toto G_1) \to (H_2\toto H_1)$ is Morita.
(concrete example: $(\Z\times\R\toto\R)\to (S^1\toto S^1)$)
\end{example}

% Ex: Abelian group bundle suspension
\begin{example}
If $(G\toto M\toto M)$ and $(H\toto N\toto N)$ are Lie 2-groupoids having only identities as arrows, as in example [int ref], a morphism $\phi : (G\toto M\toto M)\to (H\toto N\toto N)$ between them is an equivalence if and only if it is an isomorphism.
\end{example}

%%%%%%%%%%%%%%%%%%%%%%%%%%%%%%%%%%%%

Recall from section \ref{sec:gl2gpd} that the General Linear 2-Groupoid $GL_{pq}$ is the Lie 2-groupoid that has linear maps $\partial:\R^p\to\R^q$ as objects, quasi-isomorphisms $\rho:\partial\to\partial'$ as arrows, and chain homotopies $\gamma:\phi\then\phi'$ as 2-cells.
We will show that the General Linear 2-Groupoid defines, up to equivalences, the 2-stack classifying the stacky vector bundles, that we will call {\bf categorified Grassmanian}.
Write $G'_{pq}\subset G_{pq}$ for the Lie 2-subgroupoid defined by the invertible arrows $\phi$. It is clear that $(G'_{pq})_1\subset (G_{pq})_1$ is an open subset, and therefore, it is easy to see that it is well-defined.

\begin{lemma}
The inclusion $i:GL_{pq}' \to GL_{pq}$ is a surjective equivalence of Lie 2-groupoids.
\end{lemma}

\begin{proof}
The map on the objects is the identity, so E1 holds trivially.
To see that $i$ is locally Morita, let $\partial,\partial'$ be two objects in $GL_{pq}$, we have to show that
$$ \big( (GL'_{pq})_2(\partial',\partial)\toto (GL'_{pq})_1(\partial',\partial) \big) \xto{i} \big( (GL'_{pq})_2(\partial',\partial)\toto (GL'_{pq})_1(\partial',\partial) \big)$$
is Morita.
This is clearly set-theoretically fully faithful, thus, by Lemma \ref{ES-and-stFF}, we just need to show that it is smoothly essentially surjective, namely that the map
$$ t\pi_1: (GL_{pq})_2(\partial', \partial) \times_{(GL_{pq})_1(\partial', \partial)} (GL'_{pq})_1(\partial', \partial) \to (GL_{pq})_1(\partial', \partial) $$
is a surjective submersion.
And this easily follows by noting that in this particular case, $t\pi_1$ identifies with the restriction of the target map
$t: (GL_{pq})_2(\partial', \partial) \to (GL_{pq})_1(\partial', \partial)$
to the open set
$s^{-1}((GL'_{pq})_1((\partial', \partial)) \subset (GL_{pq})_2(\partial', \partial)$.
\end{proof}

%%%%%%%%%%%%%

The manifold $(GL_{pq})_0$ of objects of $GL_{p,q}$ comes equipped with a canonical filtration by the rank.
Let $U^r_{pq} \subset (GL_{pq})_0$ be the open set $\{\partial\in (GL_{pq})_0 \mid \rk\partial\geq r\}$.
Note that $U^0_{pq}=(GL_{pq})_0$, $U^1_{pq}=(GL_{pq})_0\setminus\{0\}$ and $U^r$ is empty for any $r > r_0 \coloneqq \min(p,q)$.
$$ (GL_{pq})_0 = U^0_{pq} \supset U^1_{pq} \supset \dots \supset U^r_{pq} \supset \dots \supset U^{r_0}_{pq} \supset U^{r_0+1}_{pq} = \emptyset $$

There is a canonical map $\lambda:GL_{pq} \to GL_{p+1,q+1}$ that sends a differential $\partial:\R^p\to\R^q$ to $\lambda(\partial)=\partial\oplus 1:\R^p\oplus\R\to\R^q\oplus\R$, a quasi-isomorphism $\rho$ to $\lambda(\rho)=\rho\oplus 1$, and a chain-homotopy $\gamma$ to $\lambda(\gamma)=\gamma\oplus 0$. It is easy to check that it is a well-defined smooth strict\nota{definí strict morphism? chequear} morphism.
Let us denote with $GL^r_{pq}$ to the Lie 2-groupoid that is obtained by restricting $GL_{pq}$ to $U^r_{pq}$.

\begin{lemma}
The morphism $\lambda$ gives an equivalence $\lambda:GL^r_{pq}\to GL^{r+1}_{p+1,q+1}$ for every $r$.% and its image $GL^1_{p+1,q+1}$.
%$$ \begin{tikzcd}[sep = small]
%GL_{pq} \ar[shorten = -.15em]{r}[pos=.45, sloped, below]{\sim}[pos=.45, sloped, %above]{\lambda} & GL^1_{p+1,q+1} \subset GL_{p+1,q+1}
%\end{tikzcd} $$
\end{lemma}

\begin{proof}
E1)
It is easy to see that $\lambda$ meets every orbit, let us show that the intersection is transversal.
Let $V\subset (GL_{p+1,q+1})_0$ be the open set consisting of the matrices that have non-vanishing bottom right entry $x_{pq}$.
By performing a fixed sequence of rows and columns operations, every $\eta\in (GL_{p+1,q+1})_0$ can be taken into an $F(\eta)\in\lambda(GL_{pq})$ of the form $\partial\oplus 1$ for some $\partial\in GL_{pq}$.
There is a well defined rank preserving differentiable surjective map $F : V\to\lambda((GL_{pq})_0)$
defined in an open set $(GL^1_{p+1,q+1})_0\subset V \subset (GL_{p+1,q+1})_0$
such that $F\circ F = F$.
For every $\partial\in (GL_{pq})_0$
$$ \Ker d_{\lambda(\partial)}F \oplus T_{\lambda(\partial)}\lambda(GL_{pq}) = T_{\lambda(\partial)}GL_{p+1,q+1} . $$
As $F$ preserves the rank, we have that $\Ker d_{\lambda(\partial)}F \subset T_{\lambda(\partial)}O_{\lambda(\partial)}$.

E2)
We have to show that for each pair of objects $\partial,\partial'\in (GL_{pq})_0$, the induced morphism of Lie 1-groupoids $\lambda_{\partial, \partial'}$ is Morita.
$$\big( (GL_{pq})_2(\partial',\partial) \toto (GL_{pq})_1(\partial',\partial) \big) \xto{\lambda_{\partial,\partial'}} \big( (GL_{pq})_2(\lambda(\partial'),\lambda(\partial)) \toto (GL_{pq})_1(\lambda(\partial'),\lambda(\partial))  \big)$$
To see that it is essentially surjective we can proceed(*) in a similar way to E1).
To see that $\lambda_{\partial, \partial'}$ is set-theoretically fully faithful is a simple straightforward computation. We can therefore conclude by Lemma \ref{ES-and-stFF}.
\end{proof}

\begin{coro}
By iterating $\lambda$ we get an equivalence $GL_{pq}\to GL^r_{p+r,q+r} \subset GL_{p+r,q+r}$ for every $r\geq0$.
%The General Linear 2-groupoid $GL_{pq}$ can be identified with $GL^r_{p+r,q+r} \subset GL_{p+r,q+r}$ for any $r\geq 0$.
\end{coro}

%%%%%%%%%%%%%%%%%%%%%%%%%%%%%%%%%%%%%%%%%%%%%%%%%%
%%%%%%%%%%%%%%%%%%%%%%%%%%%%%%%%%%%%%%%%%%%%%%%%%%
