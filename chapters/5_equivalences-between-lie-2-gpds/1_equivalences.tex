%
% SECTION 5.1 - EQUIVALENCES
%

When working with Lie groupoids (chapter \ref{ch:lie-gpds}), two fundamental concepts arise: Morita equivalences between them, and differentiable stacks.
These concepts were studied in some detail in chapter \ref{ch:stacks-and-syacky-vb}.
In some sense, one defines the other: a Morita equivalence is a morphism between Lie groupoids that induces an isomorphism between the corresponding stacks,
and conversely, a differentiable stack is defined as the class of a Lie groupoids modulo Morita equivalences.
In this section we explore and develop the analogous notions for Lie 2-groupoids, that of (Morita) equivalence between Lie 2-groupoids, and that of (differentiable) 2-stacks.
This is fundamental for our main theorem for we are going to classify stacky vector bundles over a fixed base, as defined in chapter 2, by morphisms of 2-stacks into the {\it generalized Grassmanian}, the 2-stack corresponding to the general linear 2-groupoid described in chapter 4.
Some approaches to these notion were developed in the literature, that we may summarize as follows: (1) an iterative approach to groupoids and morita equivalence, where equivalences are defined via pullbacks \cite{gs15},
and (2) a simplicial approach to higher groupoids, where the role of equivalences is played by the so-called hypercovers \cite{zhu09}.
For our purpose, we need to develop a more subtle theory, without the strictness of (1), but richer than (2). This is the main goal of the present chapter.

\bigskip

In section 1 we will give our original definition for (Morita) equivalences between Lie 2-groupoids, discuss some examples, and compare our approach with that of (1), building over our original results achieved in previous chapter.
Then, in section 2, we discuss the rudiments of the simplicial theory of higher Lie groupoids and hypercovers, discuss with more detail the case of 2-groupoids, and compare our definition with (2).
Finally, in section 3, we give the necessary constructions to make sense of a working definition of the category of differentiable 2-stacks.

%%%%%%%%%%%%%%%%

\section{Equivalences}

% setting notations

Recall that, given $G$ a Lie 2-groupoid, we have:
\begin{itemize}
\item an induced singular foliation $F_G\subset TG_0$ over the objects given by the orbits, and
\item for each pair of objects $x,y\in G_0$, we have a well-defined \emph{hom-groupoid},
$$G(y,x)=\big( G_2(y,x)\tto G_1(y,x) \big),$$
that is a regular Lie groupoid with abelian isotropy.
\end{itemize}
If $\phi\colon G\to H$ is a morphism of Lie 2-groupoids, then it must respect the orbits, and therefore the foliation $F_G\to F_H$, and it induces locally Lie groupoid morphisms
\[ \phi_{x,y}\colon G(y,x) \to H(\phi(y),\phi(x)). \]

% DEF: EQUIVALENCE
\begin{mydef}
A morphism of Lie 2-groupoids $\phi\colon G\to H$ is an \emph{equivalence} if it satisfies the following two conditions:
\begin{enumerate}
\item[E1)] $\phi_0\colon G_0\to H_0$ meets transversely every orbit, namely, $G_0\to H_0\to H_0/F_H$ is surjective and $T_xG_0\to N_{\phi(x)}H_0$ is surjective for every $x$;
\item[E2)] it is locally Morita, namely, for every $x,y$ the morphism of Lie 1-groupoids $\phi_{x,y}$ is Morita, as defined in \textsection \ref{sec:morita}.
\end{enumerate}
If in addition $G_0\to H_0$ is surjective submersion we say that $\phi$ is a \emph{surjective equivalence}.
\end{mydef}

We remark that if $\phi$ is an equivalence then it gives a bijection between the orbit spaces (pi0 argument explaining this), and it gives isomorphisms between the orbits of the hom-groupoids (pi1), and the isotropies of the hom-groupoids (pi2).

Let us discuss some very basic examples and we will see later more examples once we relate this definition with those in the literature.

% EX: LIE 1-GPDS
\begin{example}
If $G$ and $H$ are Lie 1-groupoids, seen as Lie 2-groupoids with trivial 2-cells, then $\phi$ is an equivalence if and only if it is Morita, as defined in \textsection\ref{sec:morita}.
  In fact, in this case, the hom-groupoids are just discrete sets, being locally Morita is the same as being set-theoretically fully faithful, and we showed in Lemma \ref{ES-and-stFF} that a Lie groupoid morphisms that meets tranversely every orbit and is set-theoretically fully faithful must be Morita.
\end{example}

% ABELIAN GROUPS / Abelian Lie 2-groups
\begin{example}
Lie 2-groups. (concrete example: $(\Z\times\R\toto\R)\to (S^1\toto S^1)$)
\end{example}

% abelian group bundle suspension
\begin{example}
$(G\toto M\toto M)\to (H\toto N\toto N)$ in this case a morphism is an equivalence if and only if it is an isomorphism.
\end{example}

%%%%%%%%%%%%%%%%%%%%%%%%%%%%%%%%%%%%

For a Lie 2-groupoid $G$, let us consider the strict Lie 2-groupoid $G'$ that is obtained from $G$ by restricting the manifold of arrows to the invertible ones.
The subset $G'_1\subset G_1$ is open.
There is an inclusion $i\colon G'\hookrightarrow G$.
For the general lineal 2-groupoid $GL_{pq}$ of a 2-vector space $\RR^p\oplus\RR^q\to *$ this morphism turns out to be a surjective equivalence.

\begin{lemma}
$i\colon GL_{pq}' \hookrightarrow GL_{pq}$ is a surjective equivalence.
\end{lemma}

\begin{proof}
The map on the objects is the identity, so it is a surjective submersion and as such it intersects tranversally every orbit of $GL_{pq}$.
To see that $i$ is locally Morita, we have to show that for any pair objects $\nu,\delta\in{GL_{pq}}_0$ the morphism of Lie 1-groupoids
\[ i_{\nu,\delta}\colon \big( {GL'_{pq}}_2(\delta,\nu)\tto {GL'_{pq}}_1(\delta,\nu) \big) \to \big( {GL_{pq}}_2(\delta,\nu)\tto {GL_{pq}}_1(\phi(\delta),\phi(\nu)) \big) \]
is Morita.
By Lemma \ref{ES-and-stFF}, it is enough to prove that $i_{\nu,\delta}$ is essentially surjective and set-theoretically fully faithful.
It is easy to see that it is set-theoreticaly fully faithful.
To see that $i_{\nu,\delta}$ is essentially surjective observe that the map
\[ t\pi_1\colon {GL_{pq}}_2(\nu, \delta) \times_{{GL_{pq}}_1(\nu, \delta)} {GL'_{pq}}_1(\nu, \delta) \to {GL_{pq}}_1(\nu, \delta) \]
is the restriction of the target map
\[ t\colon {GL_{pq}}_2(\nu, \delta) \to {GL_{pq}}_1(\nu, \delta) \]
to the open set
\[ s^{-1}({GL'_{pq}}_1(\nu, \delta)) \subset {GL_{pq}}_2(\nu, \delta) . \]
\end{proof}



\begin{lemma}
The inclusion $GL(p,q) \to GL(p+1,q+1)$ is an equivalence.
\end{lemma}

\begin{proof}
HACER
\end{proof}




