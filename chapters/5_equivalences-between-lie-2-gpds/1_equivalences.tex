%
% SECTION 1 - EQUIVALENCES
%

Let $G$, $H$ be Lie 2-groupoids, $M$ a smooth manifold and $\psi\colon M\to H_0$ a smooth map.
We say that $\psi$ meets transversally every orbit of $H$ if the map
$t\pi_1\colon H_1\times_{H_0} G_0 \to H_0$
is a surjective submersion, equivalently if the following two conditions hold.
\begin{enumerate}
\item for every $y\in H_0$ there exists a $x\in G_0$ and a $h\in H_1$ with $s(h) = \phi(x)$ and $t(h) = y$.
\item for every $x\in H_0$ the linear map induced in the normal directions $N_x\to N_{\phi(x)}$ is surjective.
\end{enumerate}
A map of Lie 2-groupoids $\phi\colon G\to H$ meets transversally every orbit of $H$ if the map on the objects $\phi_0\colon G_0\to H_0$ does.

For every $x, y\in G_0$, the $G_2(y,x)\tto G_1(y,x)$ are Lie 1-groupoids, we call them \emph{hom-groupoids}.
We say that a map of Lie 2-groupoids $\phi\colon G\to H$ is \emph{locally Morita}, or locally an equivalence, if for every pair of objects $x, y\in G_0$ the restriction to the hom-groupouid
\[ \phi_{x,y}\colon \left( G_2(y,x)\tto G_1(y,x) \right) \to \left( H_2(\phi(y),\phi(x))\tto H_1(\phi(y),\phi(x)) \right) \]
is Morita.

\begin{mydef}
$\phi$ is an \emph{equivalence} if it meets transversally every orbit and it is locally Morita.
If in addition the map on the objects $\phi_0\colon G_0\to H_0$ is a surjective submersion we say that $\phi$ is a \emph{surjective equivalence}.
\end{mydef}

\begin{example}
$GL(p,q)' \to GL(p,q)$ is a surjective equivalence.
% The map in the objects is the identity so it is a surjective equivalence.
% It is easy to see that it is set-theoreticaly locally Morita.
% To see that for fixed matrices $\delta$ and  $\nu$ the map
% $$ t\pi_1\colon GL(p,q)_2(\nu, \delta) \times_{GL(p,q)_1(\nu, \delta)} GL(p,q)_1'(\nu, \delta) \to GL(p,q)_1(\nu, \delta) $$
% is a submersion, observe that it is the restriction of the target map $t\colon GL(p,q)_2(\nu, \delta) \to GL(p,q)_1(\nu, \delta)$ to the open set $s^{-1}(GL(p,q)_1'(\nu, \delta)) \subset GL(p,q)_2(\nu, \delta)$.
\end{example}

\begin{example}
% With similar arguments as in the previuous example is easy to see that
The inclusion $GL(p,q) \to GL(p+1,q+1)$ is an equivalence.
\end{example}
