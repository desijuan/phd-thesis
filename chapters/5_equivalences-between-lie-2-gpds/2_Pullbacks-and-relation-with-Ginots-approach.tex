%
% SECTION 5.2 - PULLBACKS AND RELATION WITH GINOT'S APPROACH
%

\section{Pullbacks and relation with Ginot and Stienon's approach}

% greek letter for groupoid morphism
% latin letter for map between manifolds

% recall case of 1-gpd
In section \ref{sec:morita} we presented the pull-back of a Lie 1-groupoid $G\tto M$ through a smooth map $f: N\to M$.
It is the Lie 1-groupoid $f^*G$ over $N$ that has $G\times_{M\times M} (N\times N)$, the pull-back of manifolds of $f\times f: N\times N\to M\times M$ and the anchor map $(t,s): G\to M\times M$, as manifold of arrows \cite{dh13, mm03}.
It is important to recall that since $G\times_{M\times M} (N\times N)$ is a (good) pull-back of manifolds is not always defined \cite{dh13}.

\begin{example}
If $i: N\hookrightarrow M$ is the inclusion of a saturated submanifold and $G\toto M$ is a Lie groupoid over $M$, the pullback of $G$ through $i$ is the restriction of $G$ to $N$.
\end{example}

Let us also recall from that section that for a Lie 1-groupoid $G\toto M$ there is a foliation by the orbits, that we denote $F_G$.
If the map $f: N\to M$ is transverse the foliation $F_G$, then the pull-back $f^*G$ is well defined.
In particular, the pull-back through a submersion is always defined.

When the pull-back is defined, there is an induced canonical map $f^*G\to G$.
And it is (set-theoretic) fully faithful.
So, if $f$ is transverse to the orbits of $G$ and intersects all of them, then the canonical map is an equivalence.
Conversely, if $\phi: (G\toto M)\to (H\toto N)$ is a Morita morphism, then, as $\phi$ is essentially surjective, one can consider the pull-back $\phi^*H$ of $H$ through the map on the objects $\phi: M\to N$, and the morphism $(G\toto M)\to (\phi^*H\toto N)$ is an isomorphism.

% def of PULL-BACK OF LIE 2-GPDS
There is a similar construction for Lie 2-groupoids, which, as with Lie 1-groupoids, it is not always defined.
Let $G$ be a Lie 2-groupoid and $f: M\to G_0$ a smooth map.
If the map $f$ is transverse to the orbits of $G$, let us consider the manifolds $\psi^*G_1 \coloneqq G_1\times_{G_0\times G_0}(M\times M)$ and $\psi^*G_2 \coloneqq G_2\times_{G_0\times G_0}(M\times M)$.
The \emph{pull-back} of $G$ along $\psi$ is the Lie 2-groupoid $\psi^*G$ with objects $M$, arrows $\psi^*G_1$ and 2-cells $\psi^*G_2$.
This is fact well-defined (*).

% example
\begin{example}
Let $q: M_\U\to M$ be an open cover and $G$ a Lie 2groupoid with $G_0 = M$.
There is a Cech 2-groupoid $G_\U$ that is the pull-back of $G$ through $q$.
It has $\coprod U_i$ as objects, $\coprod G_1(U_j,U_i)$ as arrows and $\coprod G_2(U_j,U_i)$ as 2-cells. 
\end{example}

% canonical map
Given $G$ and $f:M\to G_0$ transversal to $F_G$, there is a canonical strict morphism $\phi:f^*G\to G$. This morphism is always locally an isomorphism. Therefore, if in addition, $f$ is surjective on orbit spaces, then the canonical map is an equivalence. Conversely, if $\phi$ is an equivalence (non-necesarily strict morphism), then we can pull back the second groupoid through the object map and get a canonical factorization.
$$H\to \phi^*G\to G$$

\begin{lemma}
Any equivalence can be factored as first a lax morphism that is an equivalence that is 1 on objects and second a strict morphism that is an equivalence that is locally an isomoprhism.
\end{lemma}

\begin{example}
Let $q:\tilde M\to M$ a surjective subersion, eg an open cover, and let $C,E\to M$ vector bundles.
The pullback $q^*GL(C\oplus E)$ of $GL(C\oplus E)$ through $q$ identifies canonically with the general linear 2groupoid of the pullback vector bundles: $GL(q^*C\oplus q^*E)$. In particular, the canonical map $GL(q^*C\oplus q^*E)\to GL(C\oplus E)$ is a strict morphism and an equivalence.
\end{example}

% definition of Ginot-Stienon
Pullback of Lie 2-groupoids were used in \cite{gs15} by Ginot and Stienon to define equivalences, when working with strict 2-groupoid (every arrow invertible) and strict morphisms (not lax). Concretely, a {\bf GS-equivalence} $\phi:G\to H$ between strict 2-groupoids is a strict morphism that is
\begin{enumerate}
\item[GS1)] surjective submersion on objects;
\item[GS2)] the map on the arrows $G_1\to \phi^*H_1$ is a surjective submersion; and
\item[GS3)] the pullback $ \big( G_2\toto G_1 \big) \to \big( \phi^*H_2\toto \phi^*H_1 \big) $ is Morita.
\end{enumerate}

% \begin{comment}
% They give a definition of equivalences of (strict) Lie 2-groupoids in terms of the pull-back that we defined in the previous paragraph.
% Let $\phi\colon G\to H$ a morphism of Lie 2-groupoids.
% In first place, they ask the map on the objects $\phi_0\colon G_0\to H_0$ to be a surjective submersion, so that they can consider the pullback of $H$ along $\phi_0$, that we denote with $\phi^*H$, together with the following factorization.
% \begin{equation}
% \begin{tikzcd}[sep = small]
%   G \ar{rr}{\phi} \ar{dr} &   & H \\
%   & \phi^*H \ar{ur}
% \end{tikzcd}
% \end{equation}
% For Ginot and Stienon the morphism $\phi$ is an equivalence, which they call hypercover, if in addition to $\phi_0$, the map on the arrows $G_1\to \phi^*H_1$ is also a surjective submersion, and the iduced morphism of Lie 1-groupoids $\big(G_2\tto G_1\big)\to \big(\phi^*H_2\tto \phi^*H_1\big)$ is Morita.
% \end{comment}

In the next proposition we show that our definition is indeed a generalizaation of the defintion of GS. In other words, when working with strict Lie 2-groupoids, our notion of (surjective) equivalence matches the one of Ginot and Stienon.

\begin{prop}
Let $\phi:G\to H$ a morphism of 2-groupoids such that $\phi_0$ is transverse to $F_H$. Then
$\phi$ is locally Morita if and only if the pullback $ \big( G_2\toto G_1 \big) \to \big( \phi^*H_2\toto \phi^*H_1 \big) $ is Morita.
\end{prop}

\begin{proof}
It is more or less clear that their definition, in that particular case, implies ours. The subtle part here is that condition GS3) actually implies locally Morita. But this is easy (*). The point here is that set-theoretically we have $(G_2\toto G_1)=\coprod_{x,y}(G_2(y,x)\toto G_1(y,x))$.

$$\xymatrix{  G_1 \ar[rr] \ar[rd]& & \phi^*H_1\ar[dl] \\ & G_0\times G_0 & }$$

\end{proof}


\begin{coro}
In particular, if $\phi:G\to H$ is strict and  $\phi_0$ and $G_1\to\phi^*H_1$ are surjective submersions, then $\phi$ is GS-equivalence iff it is an equivalence.
\end{coro}


%%%%%%%%%%%%%%%%%%%%%%%%%%%%%%%%%%%%%%%%%%%%%%%%%%
%%%%%%%%%%%%%%%%%%%%%%%%%%%%%%%%%%%%%%%%%%%%%%%%%%

