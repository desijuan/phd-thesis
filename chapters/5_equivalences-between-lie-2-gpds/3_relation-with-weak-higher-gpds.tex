%
% SECTION 5.3 - RELATION WITH WEAK HIGHER GPDS
%

\section{Relation with weak higher groupoids}

In the previous section we discussed the notion of equivanece in the context of strict Lie 2-groupoids, that is a somewhat less general.
The other related notion is the one of hypercover in the context of simplicial manifolds, which a priori seems more general.

Let $\Delta$ be the category of finite ordinals, whose objects are the ordered sets
$$[n] = {\{n,n-1,\dots,0\}} \quad n\geq 0$$
and its morphisms are the non-decreasing functions.
It is generated by the elemental injections $d^i:[n-1]\to [n]$ which skip the value $i\in[n]$
and the elemental surjections $s^j:[n+1]\to [n]$ which repeat the value $j\in [n]$.
% These maps verify the following relations:

A \emph{simplicial manifold} is a (contravariant) functor $X:\Delta^\text{o}\to \text{Man}$.
It  consists of a sequence of manifolds $X_n\coloneqq X([n])$ together with maps $d_i\coloneqq X(d^i):X_n\to X_{n-1}$, called face operators, and $s_j\coloneqq X(s^j):X_n\to X_{n+1}$, called degeneracies, satisfying the simplicial identities.
\begin{equation}
  \begin{alignedat}{2}
    d_id_j &= d_{j-1}d_i \quad &i &< j \\
    s_is_j &= s_{j+1}s_i \quad &i &\leq j
  \end{alignedat} \qquad\quad
  \begin{alignedat}{2}
    d_is_j &= s_{j-1}d_i \quad &i &< j \\
    d_is_j &= s_jd_{i-1} \quad &i &> j+1
  \end{alignedat} \quad
  \begin{alignedat}{2}
    d_is_j = \text{id} \quad i = j,\, j+1 \\
    & & &
  \end{alignedat}
\end{equation}
If $X$ and $Y$ are simplical manifols, a map between them $\phi:X\to Y$ is a natural transformation $\phi:X\to Y$.
It consists of a sequence of maps $\phi_n:X_n\to Y_n$ that commute with the faces and the degeneracies.

\begin{example}
In section \ref{sec:nerve-and-cohomology} we discussed briefly the nerve of a Lie groupoid, which is an example of a simplicial manifold.
\end{example}

%HIGHER STACKS
% \begin{mydef}
% Two simplicial manifolds $X$ and $Y$ are equivalent if there is a third simplicial manifold $Z$ and hypercovers $Z\to X$ and $Z\to Y$.
% A \emph{higher stack} is an equivalence class of simplicial manifolds.
% \end{mydef}

% WEAK N-GPDS
A \emph{weak Lie 2-groupoid} is a simplicial manifold $X$ that satisfies the horn-filling condition for all $n$ and such the horn-filling is unique for all $n\geq 2$, that is, in which the maps $X_n\to \Lambda_i X_n$ are surjective submersions for every $n$, and these maps are diffeomorphisms for every $n\geq 2$.

\bigskip
\noi NERVE LIE 2-GROUPOID
\bigskip

A map of between weak Lie 2-groupoids $X\to Y$ is an equivalence, also called hypercover, if $X_0\to Y_0$ and $X_1\to (X_0\times X_0)\times_{Y_0\times Y_0} Y_1$ are surjective submersions and $X_2\to M_2X\times_{M_2Y} Y_2$ is a diffeomorphism.
Here $M_2X = \Hom(\partial\Delta_2, X)$ is referred in some places as the matching space.

One problem that appears when working with weak Lie groupoids is that the matching space does not always admit a manifold structure.
An example of this is the nerve of the action groupoid $S^1\times\RR^2 \tto \RR^2$ considered as a Lie 2-groupoid with only trivial 2-cells, where $S^1$ is acting on $\RR^2$ by rotations.
The matching space
$ M_2(S^1\times\RR^2) = \{ (g_1,g_2,g_3,x)\in S^1\times S^1\times S^1\times\RR^2 \mid g_2g_1 x = g_1 x \} $
can be written as the union of $S^1\times S^1\times S^1\times \{0\}$ and $S^1\times S^1\times (\RR^2\setminus \{0\})$, so it does not admit a manifold structure.

% PROP RELACIONANDO NUESTRA DEF CON LA DE CHENCHANG
\begin{prop}
If $\phi\colon G\to H$ is a morphism of Lie 2-groupoids such that the map on the objects $\phi_0\colon G_0\to H_0$ is a surjective submersion, then $\phi$ is an equivalence if and only if $N\phi$ is an hypercover.
\end{prop}

