%
% SECTION 6.2 - THE MAIN THEOREM
%

\color{red} % EN ROJO DESDE ACÁ
%%%%%%%%%%%%%%%%%%%%%%%%%%%%%%%

\begin{thm}
  $M$ manifold, $k>0$.
  \[ VB_k[M] \quad \rightleftharpoons \quad \left\{ \text{maps of stacks} \ M\to GL_k \right\} \]
\end{thm}

\textbf{Examples:}
\begin{itemize}
  \item Manifolds: $M\tto M$.
  \item \v Cech groupoid: $\U=\{U_i\}$, \ \( M_\U = \big( \coprod U_{ji}\tto \coprod U_i \big) \).
  \item Lie groups: $G\tto *$.
  \item Group actions: $G\acts M$, \ $G\times M\tto M$.
\end{itemize}

$\phi\colon(G\tto M)\to(G'\tto M')$ morphism of Lie groupoids.

fractions \ $(G\tto M) \xleftarrow{\sim} (H\tto N) \rightarrow (G'\tto M')$.

$E\to M$ vector bundle

$\U=\{U_i\}$ trivializing open cover

$\varphi_i\colon \res{E}{U_i}\to U_i\times\RR^k$ local trivializations
\begin{equation*}
  \{\theta_{ji}\} \text{$GL_k$-cocycle:}
  \begin{cases}
    \theta_{ji}\colon U_{ji}\to GL(k,\RR)\\
    \theta_{ii}=\text{id}$, $\theta_{kj}\theta_{ji}=\theta_{ki}
  \end{cases} \qquad\qquad\qquad\qquad\qquad\qquad
\end{equation*}
Morphism of Lie groupoids \ $\theta\colon \big( \coprod U_{ji}\tto \coprod U_i \big)\to (GL_k\tto *)$

inducing a map of diff stacks
\[ (\theta, \pi)\colon
  \begin{tikzcd}[sep=small]
    (M\tto M) & \big( \coprod U_{ji}\tto \coprod U_i \big) \ar[shorten = -.15em]{l}[pos=.45, sloped, below]{\sim}[pos=.45, sloped, above]{\pi} \ar[shorten = -.15em]{r}[pos=.45]{\theta} & (GL_k\tto *)
  \end{tikzcd}
\]

\[ VB_k[M] \quad \rightleftharpoons \quad \left\{ \text{maps of stacks} \ M\to GL_k \right\} \]

%%%%%%%%%%%%%%%%%%%%%%%%%%%%%%%%%
\color{black} % EN ROJO HASTA ACÁ

\begin{thm}
Let $G\colon G_1\tto G_0$ be a Lie 1-groupoid.
For every $p,q>0$ the following corresponcence holds:
\[ VB_{pq}[G] \quad \rightleftharpoons \quad \left\{ \text{ maps of 2-stacks $G\to GL_{pq}$ } \right\} \]
\end{thm}

$G$, $H$ Lie 2-groupoids, $M$ manifold, $\psi\colon M\to H_0$  smooth map.

$\psi$ meets transversally every orbit of $H$ if:
\begin{enumerate}
\item $\forall y\in H_0$, $\exists x\in M$ and a $y\xfrom{h}\psi(x) \in H_1$.
\item $\forall x\in H_0$, $N_x\to N_{\phi(x)}$ is surjective.
\end{enumerate}

$\phi\colon G\to H$ map of Lie 2-groupoids.

\[ \phi_{x,y}\colon \left( G_2(y,x)\tto G_1(y,x) \right) \to \left( H_2(\phi(y),\phi(x))\tto H_1(\phi(y),\phi(x)) \right) \]
is Morita.

Surjective equivalence: equivalence + the map on the objects $\phi_0\colon G_0\to H_0$ is a surjective submersion.

\textbf{Remark:} For Lie 1-groupoids: equivalence = Morita morphism.

\textbf{Examples:}
\begin{itemize}
  \item $GL(p,q)' \to GL(p,q)$ is a surjective equivalence.
  \item $GL(p,q) \to GL(p+1,q+1)$ is an equivalence.
\end{itemize}

\[
  \begin{tikzcd}[sep=small]
    & \phi_0^*H \ar{dr} \\
    G \ar{ur} \ar{rr}{\phi} & & H
  \end{tikzcd}
\]
\textbf{Equivalence:} $G_1\to \phi_0^*H_1$ surjective submersion and $(G_2\tto G_1)\to (\phi_0^*H_2\tto \phi_0^*H_1)$ Morita.

simplicial manifolds in which every inner horn admits a filling and the filling is unique for $n>2$.
$\phi\colon X\to Y$ morphism of weak Lie 2-groupoids.

\textbf{Equivalence:} $X_n \to M_n X \times_{M_n Y} Y_n$ surjective submersions for $0\leq n<2$ and diffeomorpism for $n=2$.

\begin{thm}[dH-D]
  \begin{itemize}
    \item $\phi\colon G\to H$ morphisms of Lie 2-groupoids such that $\phi_0\colon G_0\to H_0$ is a surjective submersion, then \\
    \begin{center}\( \phi \ \text{equivalence} \Leftrightarrow N\phi \ \text{hypercover} \).\end{center}

    \item $G$, $H$ strict Lie 2-groupoids, $\phi\colon G\to H$ strict morphism such that $\phi_0\colon G_0\to H_0$ is a surjective submersion, then
    \begin{center}\( \phi \ \text{equivalence} \Leftrightarrow (G_2\tto G_1)\to (\phi_0^*H_2\tto \phi_0^*H_1) \ \text{Morita} \).\end{center}
  \end{itemize}
\end{thm}

Lie 2-groupoids up to equivalence:
\[
  \begin{tikzcd}[sep = small]
    G & H \ar[shorten = -.15em]{l}[pos=.45, above]{\sim} \ar[shorten = -.15em]{r}[pos=.45]{\sim} & G'
  \end{tikzcd}
\]

\textbf{Fractions:} invert the equivalences.
\[ (\psi, \phi)\colon
  \begin{tikzcd}[sep=small]
    G & H \ar[shorten = -.15em]{l}[pos=.45, sloped, below]{\sim}[pos=.45, sloped, above]{\phi} \ar[shorten = -.15em]{r}[pos=.45]{\psi} & G'
  \end{tikzcd}
\]

Two pairs $(\psi_1, \phi_1)$ and $(\psi_2, \phi_2)$ are equivalent if there is third one $(\psi_3, \phi_3)$ that sits on top of the first two.
\[
  \begin{tikzcd}
    & H_1 \ar{dl}[pos=.4, sloped, below]{\sim}[pos=.4, sloped, above]{\phi_1} \ar{dr}[pos=.4, sloped, above]{\psi_1} & \\
   G & H_3 \ar{l}[pos=.4, sloped, below]{\sim}[pos=.4, sloped, above]{\phi_3} \ar{u}[pos=.4, sloped, above]{\sim} \ar{r}[pos=.4, sloped, above]{\psi_3} \ar{d}[pos=.4, sloped, below]{\sim} & G' \\
    & H_2 \ar{ul}[pos=.4, sloped, above]{\phi_2}[pos=.4, sloped, below]{\sim} \ar{ur}[pos=.4, sloped, above]{\psi_2} &
  \end{tikzcd}
\]

\begin{itemize}
  \item Relation with the other definition of Lie groupoids (simplicial approach).
  \item Cohomology of $GL_{pq}$? $\rightarrow$ Characteristic Classes.
  \item Applications to Poisson Geometry:
  Poisson and Dirac structures integrate to groupoids and pre-groupoids respectively. The tangent spaces induce stacky vector bundles. Characteristic classes as invariants.
\end{itemize}
