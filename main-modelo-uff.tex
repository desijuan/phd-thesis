\documentclass[11pt,a4paper,oldfontcommands]{memoir}
\usepackage[utf8]{inputenc}
\usepackage[T1]{fontenc}
\usepackage[nopatch=eqnum]{microtype}
\usepackage{graphicx}
\usepackage{xcolor}
\usepackage{times}
\usepackage{pdfpages}
\usepackage[pdftex]{hyperref}
\usepackage{hypcap}
\usepackage{enumitem}

\usepackage[total={210mm,297mm},left=20mm,right=20mm,bindingoffset=10mm,top=25mm,bottom=25mm]{geometry}

%
% Packages e Ambientes
%

\usepackage{amsmath,amssymb,amsthm,mathtools,tensor,bm}
\usepackage{stmaryrd} % para \mapsfrom
\mathtoolsset{showonlyrefs}

\usepackage[english]{babel}

\OnehalfSpacing

%
% Estilos para capítulos e seções
%

\chapterstyle{madsen}
\setsecheadstyle{\Large\bfseries\sffamily\raggedright}
\setsubsecheadstyle{\large\bfseries\sffamily\raggedright}
\setsubsubsecheadstyle{\bfseries\sffamily\raggedright}

%
% Numeração
%

\pagestyle{plain}
\makepagestyle{plain}
\makeevenfoot{plain}{\thepage}{}{}
\makeoddfoot{plain}{}{}{\thepage}
\makeevenhead{plain}{}{}{}
\makeoddhead{plain}{}{}{}

%
% Package para produção do texto em latim para o exemplo do modelo --- pode (talvez deva) ser removido
%

\usepackage{lipsum}
\usepackage{kantlipsum}

%
%  Seus macros
%

\usepackage[all,2cell]{xy}
\UseAllTwocells
\usepackage{tikz-cd}

\usepackage{amsmath,amssymb,amsthm,mathtools,tensor,bm}
\usepackage{stmaryrd} % para \mapsfrom
\mathtoolsset{showonlyrefs}

\usepackage{comment}

\newtheorem{thm}{Theorem}[section]
\newtheorem{prop}[thm]{Proposition}
\newtheorem{coro}[thm]{Corolary}
\newtheorem{lemma}[thm]{Lemma}
\theoremstyle{definition}
\newtheorem{mydef}[thm]{Definition}
\newtheorem{obs}[thm]{Remark}
\newtheorem{example}[thm]{Example}
\newtheorem{examples}[thm]{Examples}
\newtheorem{q}{Question}[section]

\newcommand\NN{\mathbb N}
\newcommand\ZZ{\mathbb Z}
\newcommand\QQ{\mathbb Q}
\newcommand\RR{\mathbb R}
\newcommand\CC{\mathbb C}
\newcommand\g{\mathfrak g}
\newcommand\U{\mathcal U}
\newcommand\from{\leftarrow}
\newcommand\To{\Rightarrow}
\newcommand\From{\Leftarrow}
\newcommand\xto{\xrightarrow}
\newcommand\xfrom{\xleftarrow}
\newcommand\dashto{\dashrightarrow}
\newcommand\tto{\rightrightarrows}
\newcommand\acts{\curvearrowright}
\newcommand\res[2]{\left.#1\right|_{#2}}

\newcommand\R{\mathbb R}
\newcommand\Z{\mathbb Z}
\newcommand\toto{\rightrightarrows}
\newcommand\then{\Rightarrow}

\DeclareMathOperator{\Ker}{ker}
\DeclareMathOperator{\coker}{coker}
\DeclareMathOperator{\im}{Im}
\DeclareMathOperator{\id}{id}
\DeclareMathOperator{\rk}{rk}
\DeclareMathOperator{\Hom}{Hom}
\DeclareMathOperator{\End}{End}
\DeclareMathOperator{\Iso}{Iso}

\DeclareRobustCommand{\2cell}[5]{\xymatrix{#2 \ar@{}[r]|{\Downarrow #5}& #1 \ar@/^.7pc/[l]^{#4} \ar@/_.7pc/[l]_{#3}}}

% \usepackage[dvipsnames]{xcolor}

\newcommand{\red}{\textcolor{red}}
\newcommand{\noi}{\noindent}
\DeclareRobustCommand{\nota}[1]{\marginpar{\flushleft \tiny #1}}
\newcommand{\flechita}{\nota{\red{$\leftarrow$ Acá}}}
\newcommand{\here}{\nota{\colorbox{red}{\textbf{\large Acá!}}}}

% redefinimos \emph para que sea bold
\let\emph\relax
\DeclareTextFontCommand{\emph}{\bfseries}

% line numbers!
\usepackage[pagewise, modulo, mathlines]{lineno}
% \usepackage[pagewise, modulo, displaymath, mathlines]{lineno}
% \usepackage[displaymath, mathlines]{lineno}

\newcommand{\TITULO}{Stacky vector bundles and generalized Grassmannians}
\newcommand{\AUTOR}{Juan Desimoni}
\newcommand{\ORIENTADOR}{Matias L.~del Hoyo}


%%%%%%%%%%%%%%%%%%%%%%%%%%%%%%%%%%%%%%%%%%%%%%%%%%%%%%%%%%%%%%%%%%%%%%%%%%%%%%%%

\begin{document}

%
% Capa
%

\thispagestyle{empty}

{\sffamily\centering\Large

\includegraphics[scale=0.25]{modelo/UFF_brasao.png}

~\vspace{2cm}

Universidade Federal Fluminense

\vspace{\fill}

{\huge\TITULO}

\vspace{3.5cm}

{\LARGE\AUTOR}

\vspace{\fill}


{\large Niterói }


{\large  July 2023}


}

\clearpage

%
% Folha de rosto
%

\thispagestyle{empty}

{\sffamily\centering\Large

~\vspace{\fill}

{\huge\TITULO}

\vspace{3.5cm}

{\LARGE\AUTOR}

\vspace{3.5cm}

{\normalsize\raggedleft
\begin{minipage}{210pt}
Dissertação submetida ao Programa de Pós-Graduação
em Matemática da Universidade Federal Fluminense
como requisito parcial para a obtenção do grau de
Doutor em Matemática.
\end{minipage}
}

\vspace{3.5cm}

Advisor: \ORIENTADOR

\vspace{\fill}

{\large Niterói }


{\large  July 2023}


}

\clearpage

\thispagestyle{empty}

{\centering \includepdf{modelo/ficha_catal_modelo}}

\clearpage

\thispagestyle{empty}

{\sffamily\centering

\textbf{Dissertação de Doutorado da Universidade Federal Fluminense}

\vspace{1cm}

 por

\vspace{1cm}

\textbf{\AUTOR}

\vspace{1.5cm}

apresentada ao Programa de Pós-Graduação em Matemática como requesito parcial para a obtenção do grau de

\vspace{1.5cm}

\textbf{Doutor/Mestre em Matemática}

\vspace{1.5cm}

Título da tese:


\hrulefill

\begin{minipage}{0.8\textwidth}
\centering
\textbf{\TITULO}
\end{minipage}
~\vspace{5pt}
\hrule

\vspace{1cm}

\textit{Defendida publicamente em XX julho de 2023.}

\vspace{1cm}

{\raggedright Diante da banca examinadora composta por:}

~\vspace{5pt}

%
% Liste os nomes por ordem alfabética
\begin{tabular}{lll}
  \ORIENTADOR & UFF & Orientador\\
  Simon Chiossi & UFF & Examinador\\
  Thiago Fassarela & UFF & Examinador\\
  Henrique Bursztyn & IMPA & Examinador\\
  Cristián Ortiz & USP & Examinador\\
\end{tabular}

~\vspace{\fill}

}

\clearpage

\thispagestyle{empty}

{
{\centering\textbf{DECLARAÇÃO DE CIÊNCIA E CONCORDÂNCIA DO(A) ORIENTADOR(A)}}

~\vspace{2cm}

Autor(a) da Dissertação: \AUTOR

Data da defesa: XX/07/2023

Orientador(a): \ORIENTADOR

\vspace{2cm}

Para os devidos fins, declaro \textbf{estar ciente} do conteúdo desta \textbf{versão corrigida} elaborada em atenção às sugestões  dos membros da banca examinadora na sessão de defesa do trabalho, manifestando-me \textbf{favoravelmente} ao seu encaminhamento e publicação no \textbf{Repositório Institucional da UFF}.

~\vspace{1cm}

{\raggedright Niterói, XX/03/2023.}

~\vspace{1cm}

\begin{center}
\begin{minipage}{200pt}
\centering
	\hrule

	Nome do orientador(a)
\end{minipage}
\end{center}


}

\clearpage

\thispagestyle{empty}

{

~\vspace{\fill}

\raggedleft
Opcional para esta e aquela pessoa\\
e ainda outra pessoa\\
}

\clearpage

\thispagestyle{empty}

{

\sffamily

{\Large \centering

  AGRADECIMENTOS

}

~\vspace{1cm}

Agradeço a mim mesmo (ou a mais gente). Esta parte é opcional


Agradecimento às agências de fomento --- esta parte \textbf{NÃO É OPCIONAL!}


\textbf{Todas} as teses devem ter o seguinte agradecimento:

\begin{center}
\fbox{
\begin{minipage}{300pt}
O presente trabalho foi realizado com apoio da Coordenação de Aperfeiçoamento de Pessoal de Nível Superior - Brasil (CAPES) - Código de Financiamento 001.
\end{minipage}
}
\end{center}

No caso de bolsa CAPES, sugere-se acrescentar um agradecimento à CAPES pela bolsa. No caso de bolsas de outras agências deve ser verificada quais as regras da agência em particular. Em geral, sugere-se o seguinte: Esse trabalho foi apoiado com uma bolsa de mestrado/doutorado da/do nome da agência, número do processo da bolsa. Outra possibilidade e agradecer a agência em questão a bolsa, mencionando o número do processo ao final.

}

\clearpage

\thispagestyle{empty}

{

	\sffamily

	{\Large	\centering

		RESUMO

	}

~\vspace{1cm}

\lipsum[1-2]

}

\clearpage

\thispagestyle{empty}

{

	\sffamily

	{\Large\centering

		ABSTRACT

	}

~\vspace{1cm}

\kant[1-2]

}


\clearpage
\tableofcontents

% \linenumbers

\input{chapters}

\begin{thebibliography}{9}
% \addcontentsline{toc}{chapter}{Bibliography}

\bibitem[AAC13]{aac13}
C.~Arias Abad, M.~Crainic, \textit{Representations up to homotopy and Bott's spectral sequence for Lie groupoids}, Advances in Mathematics 248, 2013, 416--452.

\bibitem[BC04]{bc04}
J.~Baez, A.~Crans, \textit{Higher-Dimensional Algebra VI: Lie 2-Algebras}, Theory and Applications of Categories 12 (15), 2004, 492--538.

\bibitem[BT82]{bott-tu}
R.~Bott, L.~Tu, \textit{Differential forms in algebraic topology}, GTM 82, Springer, 1982.

\bibitem[BCdH16]{bcdh16}
H.~Bursztyn, A.~Cabrera, M.~del Hoyo, \textit{Vector bundles over Lie groupoids and algebroids}, Advances in Mathematics 290, 2016, 163--207.

\bibitem[GS15]{gs15}
G.~Ginot, M.~Stienon, \textit{G-gerbes, principal 2-group bundles and characteristic classes}, Journal of Symplectic Geometry 13 (4), 2015, 1001--1047.

\bibitem[GSM17]{gsm17}
A.~Gracia-Saz, R.~Mehta, \textit{VB-groupoids and representation theory of Lie groupoids}, Journal of Symplectic Geometry 15 (3), 2017, 741--783.

\bibitem[dH13]{dh13}
M.~del Hoyo, \textit{Lie Groupoids and their orbispaces}, Portugaliae Mathematica 70 (2), 2013, 161--209.

\bibitem[dHF19]{dhf19}
M.~del Hoyo, R.~L.~Fernandes, \textit{Riemannian metrics on differentiable stacks}. Mathematische Zeitschrift 292, 2019, 103--132.

\bibitem[dHO20]{dho20}
M.~del Hoyo, C.~Ortiz, \textit{Morita Equivalences of Vector Bundles}, International Mathematics Research Notices 2020 (14), 2020, 4395--4432.

\bibitem[dHS19]{dhs19}
M.~del Hoyo, D.~Stefani, \textit{The general linear 2-groupoid}, Pacific Journal of Mathematics 298, 2019, 33--57.

\bibitem[dHT19]{dht19}
M.~del Hoyo, G.~Trentinaglia, \textit{Simplicial vector bundles and representations up to homotopy}, arXiv:2109.01062.

\bibitem[Mack05]{mack05}
K.~Mackenzie, \textit{General Theory of Lie Groupoids and Lie Algebroids}, Cambridge University Press, 2005.

\bibitem[Mil74]{mil74}
J.~Milnor, J.~Stasheff, \textit{Characteristic Classes}, Princeton University Press, 1974.

\bibitem[MM03]{mm03}
I.~Moerdijk, J.~Mrcun, \textit{Introduction to foliations and Lie groupoids}, Cambridge University Press, 2003.

\bibitem[Stud19]{stud19}
F.~Studzinski, \textit{On the cohomology of representations up to homotopy of Lie groupoids}.

\bibitem[Wolf16]{wolf16}
J.~Wolfson, \textit{Descent for n-bundles}, Advances in Mathematics 288, 2016, 527--575.

\bibitem[Zhu09]{zhu09}
C.~Zhu, \textit{n-groupoids and stacky groupoids}, International Mathematics Research Notices 2009 (21), 2009, 4087--4141.

% \bibitem[]{}
% , \textit{}, , , .

\end{thebibliography}


\end{document}
